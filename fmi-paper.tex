% Template for FMI-2011 paper; to be used with:
%          fmiconf.sty - LaTeX style file, and
%          IEEEbib.bst - IEEE bibliography style file.
% --------------------------------------------------------------------------
\documentclass{article}
\usepackage{fmiconf,amsmath,epsfig,graphics}
\usepackage[utf8]{inputenc}

% Place images in the folder images
\graphicspath{{images/}}

% Example definitions.
% --------------------
\def\x{{\mathbf x}}
\def\L{{\cal L}}

\title{Einsatz von Mobile Devices mit Freihand-Gestenerkennung als Präsentationsplattform im professionellen Umfeld}

\name{Saskia Schreiber, Thomas Rehm}
\address{}

\address{Technische Hochschule Mittelhessen\\
	Friedberg, Hessen}

\begin{document}
%\ninept

\maketitle

\begin{abstract}
Seit der Ablösung des Overhead-Projektors durch Laptop und Beamer kamen viele Dienste und Technologien auf den Markt, um das Erstellen von Präsentationen zu vereinfachen. 
Die eingesetzte Hardware hat sich dagegen nicht wesentlich verändert: 
Immer noch sind Laptop und Maus (oder separate Fernbedienung) in der Regel fester Bestandteil der Ausrüstung. (Quelle?) \\
In dieser Arbeit soll untersucht werden, inwieweit sich diese Geräte durch ein mobiles Endgerät mit Gestenerkennungssoftware ersetzen lassen und ob die Eingliederung in bestehende Infrastruktur (Beamer / SmartTV) praktikabel ist.

\end{abstract}
%
\begin{keywords}
mobiles Endgerät, Gestensteuerung, Präsentation
\end{keywords}
%
\section{Einführung}
\label{sec:intro}
Durch die steigende Nutzung und Verbreitung von mobilen Endgeräten sowie vernetzten Ausgabe-Geräten (Beamer/SmartTV) 
ist eine Präsentations-Lösung im professionellen Umfeld in den letzten Jahren möglich geworden. Das Tablet oder Smartphone als Grundlage der Präsentation ist möglich, insofern  

% Am ende raus:
 \cite{C2}

% References should be produced using the bibtex program from suitable
% BiBTeX files (here: strings, refs, manuals). The IEEEbib.bst bibliography
% style file from IEEE produces unsorted bibliography list.
% -------------------------------------------------------------------------
\bibliographystyle{IEEEbib}
\bibliography{strings,refs}

\end{document}
